\documentclass{article}
\usepackage[utf8]{inputenc}
\usepackage{graphicx}
\usepackage{amsmath}


\title{Titel \\}
\author{Aron, Fredrik Lauren}
\date{October 2015}

\begin{document}

\maketitle

\section{Well-posed system}
The 2-D Shrodinger equation 
\begin{equation}\label{eq:schrodinger}
u_t = -iu_{xx} + -iu_{yy} + -iuF
\end{equation}

Performing a time-dependent coordinate transformation of the form

\begin{equation}
\begin{array}{ccc}
 x = x(\tau,\alpha,\beta), & y = y(\tau,\alpha,\beta), & t = \tau \\
 \alpha = \alpha(t,x,y), & \beta = \beta(t,x,y) & t = \tau
\end{array}
\end{equation}

This leads to the transformation

\begin{equation}\label{eq:shrodinger_transf}
u_\tau = -iu_{\alpha \alpha} \alpha^2_x -iu_\alpha \alpha_{xx} - iu_{\beta \beta} \beta^2_y -iu_\beta \beta_{yy} - u_\alpha \alpha_t - u_\beta \beta_t - iuF
\end{equation}

The coordinate transformation is representing a box with changing boundaries. The boundaries of the box are parallel to the x and y-axis, Because of this, 

\begin{equation}
\begin{array}{ccc}
 x = x(\tau,\alpha), & y = y(\tau,\beta), & t = \tau \\
 \alpha = \alpha(t,x), & \beta = \beta(t,y) & t = \tau
\end{array}
\end{equation}

and this leads to $\alpha_{xx} = \beta_{yy} = 0$. Equation \eqref{eq:shrodinger_transf} can then be simplified to 

\begin{equation}\label{eq:shrodinger_transf_2}
u_\tau = -iu_{\alpha \alpha} \alpha^2_x  - iu_{\beta \beta} \beta^2_y - u_\alpha \alpha_t - u_\beta \beta_t - iuF
\end{equation}

and in 1D 

\begin{equation}\label{eq:shordinger_1D}
u_\tau = -iu_{\alpha \alpha} \alpha^2_x  -  u_\alpha \alpha_t - iuF
\end{equation}

Consider the coordinate transformation on the form 

\begin{equation}\label{eq:x_alpha}
\alpha = h(t) x
\end{equation}

where \textit{h(t)} is a function of time. This gives $\alpha_x = h(t)$ and \eqref{eq:shordinger_1D} becomes 

\begin{equation}
u_\tau = -iu_{\alpha \alpha} h(t)^2  -  u_\alpha h'(t) - iuF
\end{equation}



\end{document}